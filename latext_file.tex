Обычный текст

\hypertarget{ux437ux430ux433ux43eux43bux43eux432ux43eux43a-1-ux433ux43e-ux443ux440ux43eux432ux43dux44f}{%
\section{Заголовок 1-го
уровня}\label{ux437ux430ux433ux43eux43bux43eux432ux43eux43a-1-ux433ux43e-ux443ux440ux43eux432ux43dux44f}}

\hypertarget{ux437ux430ux433ux43eux43bux43eux432ux43eux43a-2-ux433ux43e-ux443ux440ux43eux432ux43dux44f}{%
\subsection{Заголовок 2-го
уровня}\label{ux437ux430ux433ux43eux43bux43eux432ux43eux43a-2-ux433ux43e-ux443ux440ux43eux432ux43dux44f}}

\hypertarget{ux437ux430ux433ux43eux43bux43eux432ux43eux43a-3-ux433ux43e-ux443ux440ux43eux432ux43dux44f}{%
\subsubsection{Заголовок 3-го
уровня}\label{ux437ux430ux433ux43eux43bux43eux432ux43eux43a-3-ux433ux43e-ux443ux440ux43eux432ux43dux44f}}

\hypertarget{ux437ux430ux433ux43eux43bux43eux432ux43eux43a-4-ux433ux43e-ux443ux440ux43eux432ux43dux44f}{%
\paragraph{Заголовок 4-го
уровня}\label{ux437ux430ux433ux43eux43bux43eux432ux43eux43a-4-ux433ux43e-ux443ux440ux43eux432ux43dux44f}}

\hypertarget{ux437ux430ux433ux43eux43bux43eux432ux43eux43a-5-ux433ux43e-ux443ux440ux43eux432ux43dux44f}{%
\subparagraph{Заголовок 5-го
уровня}\label{ux437ux430ux433ux43eux43bux43eux432ux43eux43a-5-ux433ux43e-ux443ux440ux43eux432ux43dux44f}}

Заголовок 6-го уровня

(п. 3)

\begin{enumerate}
\def\labelenumi{\arabic{enumi}.}
\tightlist
\item
  Заголовок 1-го уровня
\item
  Заголовок 2-го уровня
\item
  Заголовок 3-го уровня
\item
  Заголовок 4-го уровня
\item
  Заголовок 5-го уровня
\item
  Заголовок 6-го уровня
\end{enumerate}

(п. 5)

\begin{itemize}
\tightlist
\item
  \emph{Курсив}
\item
  \textbf{Жирный}
\item
  \sout{Зачеркнутый}
\item
  \textbf{\emph{Жирный с курсивом}}
\end{itemize}

(п. 6-7)

\begin{enumerate}
\def\labelenumi{\arabic{enumi}.}
\item
  \begin{quote}
  Цитата
  \end{quote}
\item
  \begin{quote}
  \begin{quote}
  Вложенная цитата
  \end{quote}
  \end{quote}
\item
\begin{verbatim}
Выделенный абзац
\end{verbatim}
\item
  \begin{itemize}
  \tightlist
  \item
    Элемент маркированного списка
  \end{itemize}
\end{enumerate}

(п. 8)

Мой \href{https://github.com/nikyoff}{гитхаб}

\href{https://github.com/nikyoff}{\includegraphics{https://avatars.githubusercontent.com/u/67484909?s=96\&v=4}}

(п. 9-10)

\begin{itemize}
\tightlist
\item[$\boxtimes$]
  Task\#1
\item[$\square$]
  Task\#2
\item[$\boxtimes$]
  Task\#3
\item[$\square$]
  Task\#4
\end{itemize}

(п. 11)

\begin{quote}
Язык разметки текста позволяет управлять оформлением (внешним видом)
отдельных элементов текста.
\end{quote}

(п. 12)

\textbf{\emph{Управление осуществляется с помощью специальных символов
или групп символов, которые обычно называют тег.}}

(п. 13)

Пример кода
\texttt{cd\ C:\textbackslash{}root\textbackslash{}labs\textbackslash{}2}

(п. 14)

\begin{Shaded}
\begin{Highlighting}[]
\KeywordTok{const}\NormalTok{ FILE\_NAME }\OperatorTok{=} \StringTok{\textquotesingle{}README.md\textquotesingle{}}\OperatorTok{;}

\BuiltInTok{console}\OperatorTok{.}\FunctionTok{log}\NormalTok{(}\VerbatimStringTok{\textasciigrave{}Where my }\SpecialCharTok{$\{}\NormalTok{FILE\_NAME}\SpecialCharTok{\}}\VerbatimStringTok{ file?!\textasciigrave{}}\NormalTok{)}\OperatorTok{;}
\end{Highlighting}
\end{Shaded}

(п. 15)
